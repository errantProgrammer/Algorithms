\section{Right Shift}

\begin{minipage}[t]{0.2\textwidth} % Ajuste del ancho
	\begin{tabular}{| c | c | c |}
		\hline
		\textbf{P} & \textbf{Q} & \textbf{P $\gg$ Q} \\ \hline
		1010000 & 1 & 1010000\\
		\hline
		1010000 & 2& 10100\\
		\hline
		1010000 & 3 & 1010\\
		\hline
		1010000 & 4 & 101\\
		\hline
	\end{tabular}
\end{minipage}
\hfill
\begin{minipage}[c]{0.4\textwidth} % Ajuste del ancho
	\textbf{Código de ejemplo:}
	\inputminted[firstline=28, lastline=30]{cpp}{code/bitwise_operation.cpp}    
\end{minipage}
\hfill
\begin{minipage}[c]{0.25\textwidth} 
	\textbf{Ejemplo de operacion:}
	\centering
	  \[
	\begin{aligned}
		&111_2 \\
		\color{pucpRojoOscuro}{\gg} \quad &2 \\
		\hline
		&011_2 = 3
	\end{aligned}
	\]
\end{minipage}
\newline

Ver que si hacemos (\(\gg n\) ) es similar a dividir por \(2^n\).
