\section{NOT}

\begin{minipage}[t]{0.05\textwidth} % Ajuste del ancho
	\begin{tabular}{| c | c |}
		\hline
		\textbf{P}  & \textbf{$\sim$ P} \\ \hline
		0 & 1\\
		\hline
		1 & 0\\
		\hline
	\end{tabular}
\end{minipage}
\hfill
\begin{minipage}[c]{0.4\textwidth} % Ajuste del ancho
	\textbf{Código de ejemplo:}
	\inputminted[firstline=19, lastline=22]{cpp}{code/bitwise_operation.cpp}    
\end{minipage}
\hfill
\begin{minipage}[c]{0.25\textwidth} 
	\textbf{Ejemplo de operacion:}
	\centering
	\[
	\begin{aligned}
		\color{pucpRojoOscuro}{\sim} \quad &1001_2 \\
		\hline
		&0110_2 = 6
	\end{aligned}
	\]
\end{minipage}
\newline
El not lo que hace nos brinda el complemento a 2, del número, no confundir con el negador lógico "!".